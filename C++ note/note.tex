\documentclass[]{article}
\usepackage[UTF8]{ctex}
\usepackage{graphicx}
\usepackage{geometry}
 \usepackage{amssymb} \usepackage{amsmath}  \usepackage{graphicx}
 \usepackage[none]{hyphenat}
\geometry{a4paper, scale = 0.85}
\title{C++ 笔记}
\author{Kang}
\begin{document}
\maketitle
\section{复合类型}
\subsection{字符串}
    \par 字符串输入:cin读取字符串以空白((空格、制表符和换行符)来确定字符串的结束位置,所以cin在获取字符数组输入时只能读取一个单词。将其储存在数组中时,会自动添加空白字符'$\backslash$0'。每次从输入中读取一行使用getline(),通过回车键输入的换行符来确定输入的结尾。输入的参数有两种,为:  \begin{center}
    	cin.getline(name, length)
    \end{center}
    
    上述语句可以读取19个字符。此外,还可以通过cin.get()来读取,该函数的参数和getline()类似,但是读取到换行符之后不会自动读取并删除换行符,单独调用无参数的cin.get()则会自动从输入队列中读取一个字符,可以消除掉换行字符。但是C++常使用指针来处理字符串。
\subsection{标准库类型string 类}
使用string前要包含string库,string初始化方式可以为 string s4(10, 'c'),将s4初始化为十个'c'。string可以执行的操作有:
\begin{figure}[h]
	\includegraphics*[width=1\textwidth]{picture/string.png}
\end{figure}
注意.size()输出的是一个无符号数,和负数进行比较时,负数会自动的转化为$2^16=65536$(16位)的补码,例如$-32 -> 65536-32 = 65504$,然后和无符号数进行比较。在使用加法运算符进行字符串的拼接时,只需要每个加法运算符左右两侧有一个是string对象即可。
\par 基于范围的for循环: for (declaration: expression)\{statement;\},declaration会创建一个变量储存expression中的元素(直接拷贝复制),然后随着循环不断向后移动,类似于python 的 for i in...
\par .find(const string\& a)函数可以找到子字符串a在对应的string中的第一个位置的索引,如果没有此字符,那么返回string::npos。 \ 函数 .substr(size\_type pos = 0, size\_type len = npos)可以进行字符串的切割,从pos开始切割len个字符(包含pos在内)。
\par 函数decltype(a) 可以返回a的类型,可以用于定义同类型的变量,例如上述类型中的size\_type(可以看做是一个无符号整数)。
\subsection{标准库类型 类模板vector}
使用vector之前要包含库vector。定义变量格式为: vector$<$T$>$ name,定义T类型的数组name。vector定义不进行初始值的指定时,库会创建一个"值初始化"的元素初值,例如int的会初始化为0,如果元素是某种类类型,则元素由类默认初始化。vector可以调用函数push\_back(a),将元素a添加到数组的末尾。此外,vector也可以和string类似,调用size和empty函数进行判断。数组长度返回值的类型为 vector$<$T$>$ size\_type 类型。
\subsection{迭代器}
使用迭代器时需要一个容器,对容器v使用迭代器时,定义:auto b = v.begin(), e = v.end();end()函数返回的迭代器称作尾后迭代器,指向的是容器中本不存在的"尾后元素",并不是最后一个元素,是一个虚拟的位置。b和e都是一个迭代器,其类型由容器定义,为容器元素类型加iterator,常量元素则为const\_iterator,例如vector$<$int$>$::iterator和string::iterator。
\subsection{结构体简介}
结构体初始化时,也可以像数组一样初始化  \{...\}。结构体之间可以使用=直接赋值    
\subsection{枚举类型}
enum spectrum {red, orange, yellow}. 则由spectrum 所定义的变量只能在red, orange, yellow这三个符号常量中取值。如果想定义符号常量,可以将类型值spectrum省略。
\subsection{引用}
int \&变量名 = 另一个变量名(必须是对象,不能是表达式或者数值),引用类型定义时必须进行初始化。

\subsection{指针与自由储存空间}
地址运算符\& \ 间接值或者解除引用运算符*,分别代表地址和地址中储存的变量值。int* 表示指向int类型的指针,是指针不是int类型。
\\动态分配内存的语句 typeName* pointer\_name = new typeName; 其中typeName可以是对象,此时指针指向的是一个"数据对象"(非面向对象的对象,只是某个内存地址中的信息)。动态数组 int* p = new int [10];释放内存对应delete [] p,数组的长度可以是变量。
\subsection{指针、数组和指针算术}
C++将数组的名字视作首元素的地址,指针变量加一后,其数值增加 一个指向的类型的所占用的字节数(不一定是一个字节)。指针是变量,但是数组名是常量,定义指针的语句将指针变量的名字去掉即为其类型。在定义指针的语句中,*和\&互为逆运算,要看清两边的数据类型再进行赋值。例如:\begin{center}
	int (*b)[3]说明b是int(*)[3]类型的指针,指向一个int的3个元素的数组。数组名字取\&是整个数组的地址,视作一个整体。例如int a[3] = {1,2,3},则\&a的类型是int(*)[3]的。取地址后自动增加一个*,数组名字的类型也会增加一个*。
\end{center}C++在解释符号  数组或指针名[偏移量i]   时,将统一转化为*(数组或指针名字 + 偏移数i),所以指针也可以像数组名字一样访问数组。\\
\textbf{指针与字符串}cout在打印一个字符串时,并不会直接将整个字符串的元素传递,而是传递第一个字符的地址,继续打印知道遇到 $\backslash$0停止打印。因此直接传递给cout一个字符串的首地址,会自动打印这个字符串。\textbf{用引号引起的字符串,也被解释为一个数组名字,即为第一个元素的首地址}。实际上,在诸多C++表达式中,char数组名、char指针以及用双引号括起的字符串都被解释为第一个元素的地址。\\
\par 使用指针指向一个结构体时,访问成员应该使用->,不能使用.运算符。但是可以使用(*指针名字).访问。
\section{循环语句}
\par \{\}代表一个代码块,在代码块中定义的变量只是局部变量,没有代码块默认第一个语句为执行的代码块(不包括后续的语句),编译器会自动地省略缩进。
\par C++的二维数组引用:[列][行]。
\end{document}
